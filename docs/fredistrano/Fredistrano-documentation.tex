\documentclass[12pt,a4paper]{report}
\usepackage[utf8]{inputenc}
\usepackage[french] {babel}
\usepackage{ucs}
\usepackage{amsmath}
\usepackage{amsfonts}
\usepackage{amssymb}
\author{Frédéric BOLLON}
\title{\textbf{Fredistrano v0.1}\\Documentation}
\begin{document}
\maketitle
\tableofcontents
\chapter{Introduction}


\chapter{Pré-requis}
\begin{itemize}
\item Un hébergement Php5 avec le "Safe Mode" non activé. L'application n'a pas été testée sur un hégergement Php4. La fonction shell_exec doit être autorisée. Vous pouvez trouver un hébergement compatible sur http://fbollon.net.
\item Un projet Php "versionné" avec Subversion
\item Pour le déploiement sur un serveur windows, il faut installer cygwin avec rsync et un client subversion. Le chemin des disques windows devront être sous cygwin du genre /c ou /cygdrive/c
\dots
\end{itemize}




\chapter{Téléchargement}
Vous trouverez les archives des différentes versions dans le répertoire "source/Fredistrano" à cette adresse : http://www.fbollon.net/downloads

\chapter{Installation}
\begin{enumerate}
\item Décompresser l'archive : tar -xzvf fredistrano_x.x.tar.gz à la racine de votre web server ou dans un dossier de votre choix.
\item Créer la base de données à l'aide du script sql qui se trouve dans /app/config/sql/fredistrano.sql.
\item Créer les fichiers app/config.php et app/database.php en s'inspirant des fichiers config.prd.php et database.prd.php.
\end{enumerate}

\chapter{Utilisation}
\chapter{Références}
\chapter{Remerciements}

\end{document}